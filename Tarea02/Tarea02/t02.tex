\begin{enumerate}
    \item Escribe en base 10 qué número representan las siguientes cadenas de 8 bits con las distintas representaciones. (2 puntos).

    \begin{table}[H]
    \begin{tabular}{|l|l|l|l|l|l|}
    \hline
             & \begin{tabular}[c]{@{}l@{}}Entero\\ sin signo\\ (natural)\end{tabular} & \begin{tabular}[c]{@{}l@{}}Entero con\\ bit de signo\end{tabular} & \begin{tabular}[c]{@{}l@{}}Entero con\\ offset, con un\\ desplazamiento\\ de -40\end{tabular} & \begin{tabular}[c]{@{}l@{}}Complemento\\ a 1\end{tabular} & \begin{tabular}[c]{@{}l@{}}Complemento\\ a 2\end{tabular} \\ \hline
    00000000 & 0 & 0 & -40 & 255 & 0\\ \hline
    10101010 & 170 & -42 & 130 & -85 & -86\\ \hline
    11110000 & 240 & -112 & 200 & -15 & -16\\ \hline
    10000000 & 128 & 0 & 88 & -127 & -128\\ \hline
    01111111 & 127 & 127 & 87 & -128 & 127 \\ \hline
    \end{tabular}
    \end{table}

    \item Diseña e implementa las funciones que se piden en el código adjunto.
    
\end{enumerate}
